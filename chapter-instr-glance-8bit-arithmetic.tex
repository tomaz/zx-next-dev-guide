\section{8-Bit Arithmetic and Logical}

\begin{minipage}{\textwidth}

\begin{instrtable}

    \begin{instruction}{ADD A,r}
        \Symbol{\SymADD{A}{r}}
            \FlagsADDr
            \OpCode{10}{\fbox{000}}{\OCT{r}}
            \Hex{..}{1}
            \Cycles{1}{4}
            \Comment{
                \tt
                \multirow{6}{*}{
                    \begin{tabular}{ll}
                        r & \OCT{r} \\
                        \hline
                        B & 000 \\
                        C & 001 \\
                        D & 010 \\
                        E & 011 \\
                        H & 100 \\
                        L & 101 \\
                        A & 111 \\
                    \end{tabular}
                }
            }
    \end{instruction}

    \begin{instruction}{ADD A,p}
        \Symbol{\SymADD{A}{p}}
            \FlagsADDr
            \OpCode{11}{011}{101}
            \Hex{DD}{2} 
            \Cycles{2}{8}
        \SkipToOpCode
            \OpCode{10}{\fbox{000}}{\OCT{p}}
            \Hex{..}{}
            \Cycles{}{}
    \end{instruction}

    \begin{instruction}{ADD A,q} 
        \Symbol{\SymADD{A}{q}}
            \FlagsADDr
            \OpCode{11}{111}{101}
            \Hex{FD}{2}
            \Cycles{2}{8}
        \SkipToOpCode
            \OpCode{10}{\fbox{000}}{\OCT{q}}
            \Hex{..}{}
    \end{instruction}

    \begin{instruction}{ADD A,n}
        \Symbol{\SymADD{A}{n}}
            \FlagsADDr
            \OpCode{11}{\fbox{000}}{110}
            \Hex{C6}{2}
            \Cycles{2}{7}
        \SkipToOpCode
            \OpRange{n}
            \Hex{..}{}
    \end{instruction}

    \begin{instruction}{ADD A,(HL)}
        \Symbol{\SymADD{A}{(HL)}}
            \FlagsADDr
            \OpCode{10}{\fbox{000}}{110}
            \Hex{86}{1} 
            \Cycles{2}{7}
            \Comment{
                \multirow{8}{*}{
                    \tt
                    \begin{tabular}{ll}
                        p & \OCT{p} \\
                        \hline
                        B & 000 \\
                        C & 001 \\
                        D & 010 \\
                        E & 011 \\
                        IX\High & 100 \\
                        IX\Low & 101 \\
                        A & 111 \\
                    \end{tabular}
                }
            }
    \end{instruction}

    \begin{instruction}{ADD A,(IX+d)}
        \Symbol{\SymADD{A}{(IX+d)}}
            \FlagsADDr
            \OpCode{11}{011}{101}
            \Hex{DD}{3}
            \Cycles{5}{19}
        \SkipToOpCode
            \OpCode{10}{\fbox{000}}{110}
            \Hex{86}{}
        \SkipToOpCode
            \OpRange{d}
            \Hex{..}{}
    \end{instruction}
	
    \begin{instruction}{ADD A,(IY+d)}
        \Symbol{\SymADD{A}{(IY+d)}}
            \FlagsADDr
            \OpCode{11}{111}{101}
            \Hex{FD}{3}
            \Cycles{5}{19}
        \SkipToOpCode
            \OpCode{10}{\fbox{000}}{110}
            \Hex{86}{}
        \SkipToOpCode
            \OpRange{d}
            \Hex{..}{}
    \end{instruction}

    \StartWithOpCode\OpCode{}{$\uparrow$}{}\\

    \begin{instruction}{ADC A,s\See{2}}
        \Symbol{\SymADC{A}{s}}
        \FlagsADCr
        \OpCode{..}{\fbox{001}}{...} 
        \Hex{}{}
        \Cycles{}{}
        \Comment{
            \multirow{8}{*}{
                \tt
                \begin{tabular}{ll}
                    q & \OCT{q} \\
                    \hline
                    B & 000 \\
                    C & 001 \\
                    D & 010 \\
                    E & 011 \\
                    IY\High & 100 \\
                    IY\Low & 101 \\
                    A & 111 \\
                \end{tabular}
            }
        }
    \end{instruction}

    \begin{instruction}{SUB s\See{2}}
        \Symbol{\SymSUB{s}}
            \FlagsSUBr
            \OpCode{..}{\fbox{010}}{...}
    \end{instruction}

    \begin{instruction}{SBC A,s\See{2}}
        \Symbol{\SymSBC{A}{s}}
            \FlagsSBCr
            \OpCode{..}{\fbox{011}}{...}
    \end{instruction}

    \begin{instruction}{AND s\See{2}}
        \Symbol{\SymAND{s}}
            \FlagsANDr
            \OpCode{..}{\fbox{100}}{...}
    \end{instruction}

    \begin{instruction}{XOR s\See{2}}
        \Symbol{\SymXOR{s}}
            \FlagsXORr
            \OpCode{..}{\fbox{101}}{...}
    \end{instruction}

    \begin{instruction}{OR s\See{2}}
        \Symbol{\SymOR{s}}
            \FlagsORr
            \OpCode{..}{\fbox{110}}{...}
    \end{instruction}

    \begin{instruction}{CP s\See{1,2}}
        \Symbol{\SymCP{s}}
            \FlagsCPr
            \OpCode{..}{\fbox{111}}{...}
    \end{instruction}

    \Empty{}

    \begin{instruction}{INC r} 
        \Symbol{\SymINC{r}}
            \FlagsINCr
            \OpCode{00}{\OCT{r}}{\fbox{100}}
            \Hex{..}{1}
            \Cycles{1}{4}
    \end{instruction}

    \begin{instruction}{INC p}
        \Symbol{\SymINC{p}}
            \FlagsINCr
            \OpCode{11}{011}{101}
            \Hex{DD}{2}
            \Cycles{2}{8}
        \SkipToOpCode
            \OpCode{00}{\OCT{p}}{\fbox{100}}
            \Hex{..}{}
    \end{instruction}

    \begin{instruction}{INC q}
        \Symbol{\SymINC{q}}
            \FlagsINCr
            \OpCode{11}{111}{101}
            \Hex{FD}{2}
            \Hex{2}{8}
        \SkipToOpCode
            \OpCode{00}{\OCT{q}}{\fbox{100}}
            \Hex{..}{}
    \end{instruction}

    \begin{instruction}{INC (HL)}
        \Symbol{\SymINC{(HL)}}
            \FlagsINCr
            \OpCode{00}{110}{\fbox{100}}
            \Hex{34}{1}
            \Cycles{3}{11}
    \end{instruction}

    \begin{instruction}{INC (IX+d)}
        \Symbol{\SymINC{(IX+d)}}
            \FlagsINCr
            \OpCode{11}{011}{101}
            \Hex{DD}{3}
            \Cycles{6}{23}
        \SkipToOpCode
            \OpCode{00}{110}{\fbox{100}}
            \Hex{34}{}
        \SkipToOpCode
            \OpRange{d}
    \end{instruction}

    \begin{instruction}{INC (IY+d)}
        \Symbol{\SymINC{(IY+d)}}
            \FlagsINCr
            \OpCode{11}{111}{101}
            \Hex{FD}{3}
            \Cycles{6}{23}
        \SkipToOpCode
            \OpCode{00}{110}{\fbox{100}}
            \Hex{34}{}
        \SkipToOpCode
            \OpRange{d}
    \end{instruction}

    \StartWithOpCode\OpCode{}{}{$\uparrow$} \\

    \begin{lastinstruction}{DEC m\See{3}}
        \Symbol{\SymDEC{m}}
            \FlagsDECr
            \OpCode{..}{...}{\fbox{101}}
    \end{lastinstruction}

\end{instrtable}

\begin{notestable}
    \NoteItem{\See{1}YF and XF flags are copied from the operand {\tt s}, not the result {\tt A-s}}

    \NoteItem{\See{2}{\tt s} is any of {\tt r}, {\tt p}, {\tt q}, {\tt n}, {\tt (HL)}, {\tt (IX+d)}, {\tt (IY+d)} as shown for {\tt ADD}. Replace \fbox{{\tt 000}} in the {\tt ADD} set above. Ts also the same}

    \NoteItem{\See{3}{\tt m} is any of {\tt r}, {\tt p}, {\tt q}, {\tt n}, {\tt (HL)}, {\tt (IX+d)}, {\tt (IY+d)} as shown for {\tt INC}. Replace \fbox{{\tt 100}} with \fbox{{\tt 101}} in opcode. Ts also the same}

    \NoteItem{\See{4}PV set if value was {\tt \$7F} before incrementing}

    \NoteItem{\See{5}PV set if value was {\tt \$80} before decrementing}
\end{notestable}

\end{minipage}
