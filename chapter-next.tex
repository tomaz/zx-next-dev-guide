\chapter{ZX Spectrum Next}

% █████████████████████████████████████████████████████████████████████████
% █░░░░░░██████████░░░░░░█░░░░░░░░░░░░░░█░░░░░░░░██░░░░░░░░█░░░░░░░░░░░░░░█
% █░░▄▀░░░░░░░░░░██░░▄▀░░█░░▄▀▄▀▄▀▄▀▄▀░░█░░▄▀▄▀░░██░░▄▀▄▀░░█░░▄▀▄▀▄▀▄▀▄▀░░█
% █░░▄▀▄▀▄▀▄▀▄▀░░██░░▄▀░░█░░▄▀░░░░░░░░░░█░░░░▄▀░░██░░▄▀░░░░█░░░░░░▄▀░░░░░░█
% █░░▄▀░░░░░░▄▀░░██░░▄▀░░█░░▄▀░░███████████░░▄▀▄▀░░▄▀▄▀░░███████░░▄▀░░█████
% █░░▄▀░░██░░▄▀░░██░░▄▀░░█░░▄▀░░░░░░░░░░███░░░░▄▀▄▀▄▀░░░░███████░░▄▀░░█████
% █░░▄▀░░██░░▄▀░░██░░▄▀░░█░░▄▀▄▀▄▀▄▀▄▀░░█████░░▄▀▄▀▄▀░░█████████░░▄▀░░█████
% █░░▄▀░░██░░▄▀░░██░░▄▀░░█░░▄▀░░░░░░░░░░███░░░░▄▀▄▀▄▀░░░░███████░░▄▀░░█████
% █░░▄▀░░██░░▄▀░░░░░░▄▀░░█░░▄▀░░███████████░░▄▀▄▀░░▄▀▄▀░░███████░░▄▀░░█████
% █░░▄▀░░██░░▄▀▄▀▄▀▄▀▄▀░░█░░▄▀░░░░░░░░░░█░░░░▄▀░░██░░▄▀░░░░█████░░▄▀░░█████
% █░░▄▀░░██░░░░░░░░░░▄▀░░█░░▄▀▄▀▄▀▄▀▄▀░░█░░▄▀▄▀░░██░░▄▀▄▀░░█████░░▄▀░░█████
% █░░░░░░██████████░░░░░░█░░░░░░░░░░░░░░█░░░░░░░░██░░░░░░░░█████░░░░░░█████
% █████████████████████████████████████████████████████████████████████████


\ChapterTOC[]

\pagebreak
\thispagestyle{plain} % use toc style without headers for this explanation page, it better matches chapter start page

With increased CPU speeds, more memory, better graphics, hardware sprites and tiles, to mention just some of the most obvious, ZX Spectrum Next is an exciting platform for the retro programmer.

This chapter represents the bulk of the book. Each topic is discussed in its own section. While the sections are laid out in order - later sections sometimes rely on, or refer to the topics discussed earlier, there's no need to go through them in an orderly fashion. Each section should be quite usable by itself as well. All topics discussed elsewhere are referenced, so you can quickly jump there if needed. If using PDF you can click on the section number to go straight to it. With a printed book though, turning the pages gives you a chance to land on something unrelated, but equally interesting, thus learning something new almost by accident.

One more thing worth mentioning, before leaving you on to explore, are ports and Next registers. You will find the full list in the next section. But that's just a list with a couple of examples on how to read and write them. Still, many ports and registers are described in detail at the end of the sections in which they are first mentioned. Those registers that are relevant for multiple topics are described in the first section they are mentioned in, and then referenced from other sections. Additionally, each port and register that's described in detail, has the reference to that section in the list on the following pages as a convenience. I thought for a while about how to approach this. One way would be to describe the ports and references in a single section, together with their list, and then only reference them elsewhere. But ultimately decided on the format described above for two reasons: descriptions are not comprehensive, only relevant ports and registers have this ``honour''. And secondly, I wanted to keep all relevant material as close together as possible.

\pagebreak
\pagestyle{clean} % all subsequent pages should use clean style with full header
