\newcommand*{\PRINTED}{}	% comment this line for generating online variant

% ░█▀▀█ ▒█▀▀█ ▒█▀▀▀█ ▒█░▒█ ▀▀█▀▀ 
% ▒█▄▄█ ▒█▀▀▄ ▒█░░▒█ ▒█░▒█ ░▒█░░ 
% ▒█░▒█ ▒█▄▄█ ▒█▄▄▄█ ░▀▄▄▀ ░▒█░░

% general document info for simpler reuse
\newcommand{\AuthorName}{Toma\v{z}}
\newcommand{\AuthorNameSurname}{\AuthorName ~Kragelj}
\newcommand{\BookTitleMain}{ZX Spectrum Next}
\newcommand{\BookTitleSub}{Assembly Developer Guide}
\newcommand{\BookTitle}{\BookTitleMain~\BookTitleSub}
\newcommand{\BookKeywords}{zx,next,spectrum,retro,documentation,manual,guide,assembly,language,programming}


% ▒█▀▀█ ▒█▀▀▀ ▒█░░▒█ ▀█▀ ▒█▀▀▀█ ▀█▀ ▒█▀▀▀█ ▒█▄░▒█ ▒█▀▀▀█ 
% ▒█▄▄▀ ▒█▀▀▀ ░▒█▒█░ ▒█░ ░▀▀▀▄▄ ▒█░ ▒█░░▒█ ▒█▒█▒█ ░▀▀▀▄▄ 
% ▒█░▒█ ▒█▄▄▄ ░░▀▄▀░ ▄█▄ ▒█▄▄▄█ ▄█▄ ▒█▄▄▄█ ▒█░░▀█ ▒█▄▄▄█

\newcommand{\LatestYear}{2021}
\newcommand{\LatestMonthName}{September}
\newcommand{\LatestMonth}{09}
\newcommand{\LatestDay}{15}

% revision name is nicely structured variant with named month
\newcommand{\LatestRevisionName}{\LatestDay~\LatestMonthName~\LatestYear}

% revision date is "reverse domain" type that can be easily sorted in various computer lists; also suitable for git tags
\newcommand{\LatestRevisionDate}{\LatestYear-\LatestMonth-\LatestDay}


% ▒█▀▀▀ ▒█░░░ ▒█▀▀▀ ▒█▀▄▀█ ▒█▀▀▀ ▒█▄░▒█ ▀▀█▀▀ ▒█▀▀▀█ 
% ▒█▀▀▀ ▒█░░░ ▒█▀▀▀ ▒█▒█▒█ ▒█▀▀▀ ▒█▒█▒█ ░▒█░░ ░▀▀▀▄▄ 
% ▒█▄▄▄ ▒█▄▄█ ▒█▄▄▄ ▒█░░▒█ ▒█▄▄▄ ▒█░░▀█ ░▒█░░ ▒█▄▄▄█

% definitions based on whether PDF is generated for book or online.
\newcommand{\email}[3]{\ifdefined\PRINTED{\tt #1@#2.#3}\else{\tt #1 AT #2 DOT #3}\fi}

% creates minitoc without any headers; parameters:
% - (optional) #1 boolean; if star present, no styling will be applied after TOC, otherwise default styling will be applied (which is the default)
% - (optional) #1 boolean; if present, chapter TOC is generated, otherwise not (default will create TOC)
\NewDocumentCommand{\ChapterTOC}{ o o }{
	\pagestyle{nosectionmarker}

	% if #2 is present (aka not "[]"), render TOC.
	\IfValueF{#2}{
		\etocsettocstyle{}{}	% no title
		\localtableofcontents
	}
	\thispagestyle{plain}	% simple TOC pages without headers
	
	% if #1 is present (aka not "[]"), set page style to clean
	\IfValueF{#1}{\pagestyle{clean}}
}

% we can use this for pages that are intentionally left blank
\newcommand{\IntentionallyEmpty}{
	\mbox{}
	\vfill
	\begin{center}
	This page intentionally left empty
	\end{center}
	\vfill
	\mbox{}
}


% ▒█▀▀█ ▒█▀▀▀█ ▒█░░░ ▒█▀▀▀█ ▒█░▒█ ▒█▀▀█ ▒█▀▀▀█ 
% ▒█░░░ ▒█░░▒█ ▒█░░░ ▒█░░▒█ ▒█░▒█ ▒█▄▄▀ ░▀▀▀▄▄ 
% ▒█▄▄█ ▒█▄▄▄█ ▒█▄▄█ ▒█▄▄▄█ ░▀▄▄▀ ▒█░▒█ ▒█▄▄▄█

\definecolor{PrintableLightGray}{rgb}{0.85, 0.85, 0.85}
\definecolor{PrintableDarkGray}{rgb}{0.5, 0.5, 0.5}


% ▒█▀▀▀█ ▒█░▒█ ▒█▀▀▀█ ▒█▀▀█ ▀▀█▀▀ ▒█░▒█ ░█▀▀█ ▒█▄░▒█ ▒█▀▀▄ ▒█▀▀▀█ 
% ░▀▀▀▄▄ ▒█▀▀█ ▒█░░▒█ ▒█▄▄▀ ░▒█░░ ▒█▀▀█ ▒█▄▄█ ▒█▒█▒█ ▒█░▒█ ░▀▀▀▄▄ 
% ▒█▄▄▄█ ▒█░▒█ ▒█▄▄▄█ ▒█░▒█ ░▒█░░ ▒█░▒█ ▒█░▒█ ▒█░░▀█ ▒█▄▄▀ ▒█▄▄▄█

% less tall slash
\newcommand\scslash{\stretchrel*{$/$}{\textsc{e}}}

% couple shorthands that can be used throughout the document
\newcommand{\Deg}{\textsuperscript{o}}
\newcommand{\ddd}{\makebox[1em][c]{.\hfil.\hfil.}}
\newcommand{\See}[1]{\textsuperscript{#1}}
\newcommand{\UNDOC}{\textnormal{\textsuperscript{**}}}
\newcommand{\ZXN}{\textnormal{\textsuperscript{ZX}}}
\newcommand{\ZXNS}{\tiny\textnormal{\textsuperscript{ZX}}}
\newcommand{\High}{\textsubscript{h}}
\newcommand{\Low}{\textsubscript{l}}

% instruction flags definitions - for unified formatting
\newcommand{\FS}{$\updownarrow$} % standard effect
\newcommand{\FN}{-}				% no effect
\newcommand{\FU}{?}				% unknown effect
\newcommand{\FX}{$\bullet$}		% special case
\newcommand{\FPV}{VF}			% PV=Overflow
\newcommand{\FPP}{PF}			% PV=Parity

% PortLink is short link to port with description and address, suitable for inline use, while PortReference is longer reference with chapter title and section number mostly used for register lists when referring to previously described register
\newcommand{\PortLink}[2]{{\small \textbf{#1} \MemAddr{#2}}}
\newcommand{\PortReference}[2]{
	\vspace*{-2ex}
	See description under #2 chapter, section \ref{#1}.
}

% other commonly used definitions
\newcommand{\MemAddr}[1]{{\tt \$#1}}


% ▒█▀▀█ ▒█▀▀▀█ ▒█▀▄▀█ ▒█▀▄▀█ ▒█▀▀▀█ ▒█▄░▒█   ▒█▀▀▄ ▒█▀▀█ ░█▀▀█ ▒█░░▒█ ░█▀▀█ ▒█▀▀█ ▒█░░░ ▒█▀▀▀ ▒█▀▀▀█ 
% ▒█░░░ ▒█░░▒█ ▒█▒█▒█ ▒█▒█▒█ ▒█░░▒█ ▒█▒█▒█   ▒█░▒█ ▒█▄▄▀ ▒█▄▄█ ▒█▒█▒█ ▒█▄▄█ ▒█▀▀▄ ▒█░░░ ▒█▀▀▀ ░▀▀▀▄▄ 
% ▒█▄▄█ ▒█▄▄▄█ ▒█░░▒█ ▒█░░▒█ ▒█▄▄▄█ ▒█░░▀█   ▒█▄▄▀ ▒█░▒█ ▒█░▒█ ▒█▄▀▄█ ▒█░▒█ ▒█▄▄█ ▒█▄▄█ ▒█▄▄▄ ▒█▄▄▄█

% horizontal arrows of arbitrary size (lehgth specified through parameter)
\newcommand{\RArrow}[1]{\parbox{#1}{\tikz{
	\draw[->,line width=0.5pt](0,0)--(#1,0);
}}}
\newcommand{\LArrow}[1]{\parbox{#1}{\tikz{
	\draw[<-,line width=0.5pt](0,0)--(#1,0);
}}}

% horizontal arrows with vertical line on the arrow side; parameters:
% - mandatory horizontal line width
% - optional any additional horizontal line styles (dotted, dashed etc)
% - optional vertical line height (divided by 2), default 0.1
% - optional arrow scale, default 1
% - optional line width, default 0.5pt
\NewDocumentCommand{\RArrowLine}{ m O{} O{0.1} O{1.6} O{0.5pt} }{\parbox{#1}{\tikz{
	\draw[-{>[scale=#4,length=#4,width=#4]},line width=#5,#2](0,0)--(#1,0);
	\draw[line width=#5](#1,-#3)--(#1,#3);
}}}
\NewDocumentCommand{\LArrowLine}{ m O{} O{0.1} O{1.6} O{0.5pt} }{\parbox{#1}{\tikz{
	\draw[line width=#5](0,-#3)--(0,#3);
	\draw[-{>[scale=#4,length=#4,width=#4]},line width=#5,#2](#1,0)--(0,0);
}}}


% ▒█▀▀█ ▒█░▒█ ▒█▀▀▀█ ▀▀█▀▀ ▒█▀▀▀█ ▒█▀▄▀█ ▀█▀ ▒█▀▀▀█ ░█▀▀█ ▀▀█▀▀ ▀█▀ ▒█▀▀▀█ ▒█▄░▒█ ▒█▀▀▀█ 
% ▒█░░░ ▒█░▒█ ░▀▀▀▄▄ ░▒█░░ ▒█░░▒█ ▒█▒█▒█ ▒█░ ░▄▄▄▀▀ ▒█▄▄█ ░▒█░░ ▒█░ ▒█░░▒█ ▒█▒█▒█ ░▀▀▀▄▄ 
% ▒█▄▄█ ░▀▄▄▀ ▒█▄▄▄█ ░▒█░░ ▒█▄▄▄█ ▒█░░▒█ ▄█▄ ▒█▄▄▄█ ▒█░▒█ ░▒█░░ ▄█▄ ▒█▄▄▄█ ▒█░░▀█ ▒█▄▄▄█

% define the style for lstlisting
\lstdefinestyle{CodeStyle}{
	basicstyle=\ttfamily\small,
	commentstyle=\color{PrintableDarkGray},
	columns=flexible,
	tabsize=4,
	numbers=left,
	numberstyle=\ttfamily\tiny,
	numbersep=1.8em,
	morecomment=[l]{;},
	moredelim=[is][\rmfamily\itshape]{|}{|},	% any text within |...| will be roman/italic
	literate={&}{\$}1							% replace `&` with `$` (to avoid syntax higlight issues)
			{Band}{\&~}1						% replace `Band` with `&` (to allow bitwise and output without & being consumed for $ from previous rule)
			{Bor}{|~}1							% replace `Bor' with `|` (to allow bitwise or output without | being consumed for roman/italic from moredelim rule above)
}

% define default settings for all tcolorbox instances
\tcbset{
	arc=4pt,			% radius for rounded corners
	boxrule=0pt,		% no frame around the box
	boxsep=1.3ex,		% add some spacing between frame and content
	left=0ex,			% left side should be flush with content
	right=0ex,			% right side should be flush with content
	top=-1.5ex,			% less spacing between frame and start of content
	bottom=-1.5ex,		% less spacing between end of content and frame
	pad at break=1pt,	% leave some small spacing between content and frame when page break occurs
	before skip=2ex,
	after skip=3ex,
	colback=PrintableLightGray,
	colframe=PrintableLightGray,
	enhanced,			% only use rounded corners on top and bottom part, not between page breaks
	breakable,			% allow breaking tcolorbox to multiple pages
	listing only,		% only show listing
	listing options={style=CodeStyle},
}
